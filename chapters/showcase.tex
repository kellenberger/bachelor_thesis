This is a showcase

\clearpage
\pagestyle{empty}
\begin{landscape}
\begin{table}[p]
\begin{tabular}{ | m{10cm} | m{10cm} | }
  \hline
  Uncorrupted Input & Output \\
  \hline
  {\begin{lstlisting}[style=table]
  public final void enable() {
    FileConfiguration config = new FileConfiguration(NoLagg.plugin);
    config.load();
    this.enable(config);
    config.save();
  }
  \end{lstlisting}} &
  {\begin{lstlisting}[style=table]
  public final void enable() {
   FileConfiguration config = new FileConfiguration(NoLagg.plugin);
   config.load();
   this.enable(config);
   config.save();
  }
  \end{lstlisting}} \\
  \hline
  {\begin{lstlisting}[style=table]
  private Vec4 parseVec4(String s) {
   Scanner snr = new Scanner(s);
   Vec4 res = new Vec4();
   res.x = Float.parseFloat(snr.next());
   res.y = Float.parseFloat(snr.next());
   res.z = Float.parseFloat(snr.next());
   res.w = Float.parseFloat(snr.next());
   return res;
  }
  \end{lstlisting}} &
  {\begin{lstlisting}[style=table]
  private Vec4 parseVec4(String s) {
   Scanner snr = new Scanner(s);
   Vec4 res = new Vec4();
   res.x = Float.parseFloat(snr.next());
   res.y = Float.parseFloat(snr.next());
   res.$y$ = Float.parseFloat(snr.next());
   res.w = Float.parseFloat(snr.next());
   return res;
  }
  \end{lstlisting}} \\
  \hline
\end{tabular}
\caption{Example}
\label{uncorrupted_showcase_table}
\end{table}

\begin{table}[p]
\begin{tabular}{ | m{10cm} | m{10cm} | }
  \hline
  Brackets Input & Output \\
  \hline
  {\begin{lstlisting}[style=table]
  int pop(int numBits) $_$
   int i = getLeadingAsInt(numBits);
   truncate(numBits);
   return i;
  }
  \end{lstlisting}} &
  {\begin{lstlisting}[style=table]
  int pop(int numBits) ^{^
   int i = getLeadingAsInt(numBits);
   truncate(numBits);
   return i;
  }
  \end{lstlisting}} \\
  \hline
  {\begin{lstlisting}[style=table]
  protected void outlineShape(Graphics graphics, Rectangle bounds) {
   PointList pl = setupPoints$_$bounds);
   graphics.drawPolygon(pl);
   int add = graphics.getLineWidth() / 2;
   graphics.drawOval(new Rectangle(ovalX, ovalY, ovalD + add, ovalD + add));
  }
  \end{lstlisting}} &
  {\begin{lstlisting}[style=table]
  protected void outlineShape(Graphics graphics, Rectangle bounds) {
   PointList pl = setupPointsbounds$($);
   graphics.drawPolygon(pl);
   int add = graphics.getLineWidth() / 2;
   graphics.drawOval(new Rectangle(ovalX, ovalY, ovalD + add, ovalD + add));
  }
  \end{lstlisting}} \\
  \hline
\end{tabular}
\caption{Example}
\label{brackets_showcase_table}
\end{table}

\begin{table}[p]
\begin{tabular}{ | m{10cm} | m{10cm} | }
  \hline
  Semicolons Input & Output \\
  \hline
  {\begin{lstlisting}[style=table]
  @Override
  public Method run() {
   try {
    final Method mtd = clazz.getMethod("writeReplace")$_$
    mtd.setAccessible(true);
    return mtd;
   } catch (NoSuchMethodException e) {}
   return null;
  }
  \end{lstlisting}} &
  {\begin{lstlisting}[style=table]
  @Override
  public Method run() {
   try {
    final Method mtd = clazz.getMethod("writeReplace")^;^
    mtd.setAccessible(true);
    return mtd;
   } catch (NoSuchMethodException e) {}
   return null;
  }
  \end{lstlisting}} \\
  \hline
  {\begin{lstlisting}[style=table]
  public void test_hashCode() {
   ExternalIdWithDates d1a = ExternalIdWithDates.of(IDENTIFIER, VALID_FROM, VALID_TO);
   ExternalIdWithDates d1b = ExternalIdWithDates.of(IDENTIFIER, VALID_FROM, VALID_TO);
   assertEquals(d1a.hashCode(), d1b.hashCode())$_$
  }
  \end{lstlisting}} &
  {\begin{lstlisting}[style=table]
  public void test_hashCode() {
   ExternalIdWithDates d1a = ExternalIdWithDates.of(IDENTIFIER, VALID_TO);
   ExternalIdWithDates d1b = ExternalIdWithDates.of(IDENTIFIER, VALID_TO);
   assertEquals(d1a.hashCode(), d1b.hashCode());
   $assertEquals(d1a.hashCode(), d1b.hashCode());$
  }
  \end{lstlisting}} \\
  \hline
\end{tabular}
\caption{Example}
\label{semicolon_showcase_table}
\end{table}

\begin{table}[p]
\begin{tabular}{ | m{10cm} | m{10cm} | }
  \hline
  Variable Input & Output \\
  \hline
  {\begin{lstlisting}[style=table]
  private boolean validateOrder(InteractionOperand interactionOperand) {
   orderedFragments = $interactionOpernd$.getFragments();
   computeConstraints();
   return reorderFragmentsInAValidTrace();
  }
  \end{lstlisting}} &
  {\begin{lstlisting}[style=table]
  private boolean validateOrder(InteractionOperand interactionOperand) {
   orderedFragments = ^interactionOperand^.getFragments();
   computeConstraints();
   return reorderFragmentsInAValidTrace();
  }
  \end{lstlisting}} \\
  \hline
  {\begin{lstlisting}[style=table]
  @Override
  public void mouseReleased(MouseEvent e) {
   popup.setVisible(false);
   String colorText = "RGB = " + buttonColor.getRed() + ", " + buttonColor.getBreen() + ", " + buttonColor.getBlue();
   this.setText($colrText$);
   this.firePropertyChange(COLOR_CHANGE, previousColor, buttonColor);
  }
  \end{lstlisting}} &
  {\begin{lstlisting}[style=table]
  @Override
  public void mouseReleased(MouseEvent e) {
   popup.setVisible(false);
   String colorText = "RGB = " + buttonColor.getRed() + ", " + buttonColor.getBreen() + ", " + buttonColor.getBlue();
   this.setText($colrText$);
   this.firePropertyChange(COLOR_CHANGE, previousColor, buttonColor);
  }
  \end{lstlisting}} \\
  \hline
\end{tabular}
\caption{Example}
\label{variable_showcase_table}
\end{table}

\begin{table}[p]
\begin{tabular}{ | m{10cm} | m{10cm} | }
  \hline
  Return type Input & Output \\
  \hline
  {\begin{lstlisting}[style=table]
  @Override
  public $void$ toString() {
   if (eIsProxy()) return super.toString();
   StringBuffer result = new StringBuffer(super.toString());
   result.append(" (name: ");
   result.append(name);
   result.append(')');
   return result.toString();
  }
  \end{lstlisting}} &
  {\begin{lstlisting}[style=table]
  @Override
  public ^String^ toString() {
   if (eIsProxy()) return super.toString();
   StringBuffer result = new StringBuffer(super.toString());
   result.append(" (name: ");
   result.append(name);
   result.append(')');
   return result.toString();
  }
  \end{lstlisting}} \\
  \hline
  {\begin{lstlisting}[style=table]
  @Override
  public $void$ evaluate(final Double...ts) {
   Validate.isTrue(ts.length == 2);
   final double tau = ts[0];
   final double s = ts[1];
   final double t = maturity - tau;
   final double temp = vol * Math.pow(s, beta) * localVol.getVolatility(t, s);
   return -0.5 * temp * temp;
  }
  \end{lstlisting}} &
  {\begin{lstlisting}[style=table]
  @Override
  public $Validate$ evaluate(final Double...ts) {
   $>$final double tau = ts[0];
   $>$Validate.isTrue(ts.length == 2);
   final double s = ts[1];
   final double t = maturity - tau;
   final double temp = vol * Math.pow(s, beta) * localVol.getVolatility(t, s);
   return -0.5 * temp * temp;
  }
  \end{lstlisting}} \\
  \hline
\end{tabular}
\caption{Example}
\label{return_type_showcase_table}
\end{table}

\begin{table}[p]
\begin{tabular}{ | m{10cm} | m{10cm} | }
  \hline
  Switch Input & Output \\
  \hline
  {\begin{lstlisting}[style=table]
  protected void disposeElementInfo(Object element, ElementInfo info) {
   if (info instanceof ResourceSetInfo) {
    $>$resourceSetInfo.dispose();
    $>$ResourceSetInfo resourceSetInfo = (ResourceSetInfo) info;
   }
   super.disposeElementInfo(element, info);
  }
  \end{lstlisting}} &
  {\begin{lstlisting}[style=table]
  protected void disposeElementInfo(Object element, ElementInfo info) {
   if (info instanceof ResourceSetInfo) {
    ^>^ResourceSetInfo resourceSetInfo = (ResourceSetInfo) info;
    ^>^resourceSetInfo.dispose();
   }
   super.disposeElementInfo(element, info);
  }
  \end{lstlisting}} \\
  \hline
  {\begin{lstlisting}[style=table]
  private void resolveEntry(Entry < K, T > entry) {
   $>$entry.isResolved = true;
   $>$resolved.add(entry);
   resolved(entry);
  }
  \end{lstlisting}} &
  {\begin{lstlisting}[style=table]
  private void resolveEntry(Entry < K, T > entry) {
   $>$entry.isResolved = true;
   $>$resolved.add(entry);
   resolved(entry);
  }
  \end{lstlisting}} \\
  \hline
\end{tabular}
\caption{Example}
\label{switch_showcase_table}
\end{table}
\end{landscape}
\clearpage
