\section{A Bit of History}

With the emergence of word processing programs and the following digitalization of text documents, spelling checkers and correctors have become a common helper in our everyday lives. However, the research on this topic has begun much earlier \cite{program_check_correction}.

The original motivation for a spelling checker was to correct input errors in databases. For example, in \cite{data_correction} the authors aim to correct names, dates and places in a genealogical database. This is done by computing the frequency of trigrams (three sequential characters) in the source text and based on that the probability of a character given some context, i.e. its adjacent characters. Erroneous words are also found by looking at its trigrams. If a word consist of a number of unusual character combinations, it is probably spelled wrongly. However, it is easy to see that this method is not very useful for rare, foreign expressions like "doppelg\"{a}nger" and for typos with high probabilities of being correct.

These problems were solved by using dictionaries to look for spelling errors \cite{program_check_correction}. A dictionary is a list of correctly spelled words which can optionally be extended by the user. For every word, the program checks if it is part of the dictionary. If it is, then it is spelled correctly, otherwise there is an error. Of course this method is not perfect either, for example if "know" is misspelled as "now", no error is indicated even though it could be concluded from the context that a verb is expected in this place.

\section{Automatic Correction of Source Code}
