\section{Naming Conventions}

Vector and matrix variables are always written in bold, vectors having lowercase names, matrices uppercase ones. For example:

\begin{equation*}
  \mathbf{a}_t, \mathbf{B}
\end{equation*}

If no other indication of the nature of the variable is given, it is a vector respectively a matrix of learnt parameters. These variables are often named \(\mathbf{W}\) for matrices and \(\mathbf{v}\) for vectors.

\bigskip

The individual elements of a vector are not written in bold and are indexed with a subscript.

\begin{equation*}
  \mathbf{a} = \begin{pmatrix} a_1 \\ ... \\ a_n \end{pmatrix}
\end{equation*}

If the vector already has a subscript, the index of the element is added as an additional subscript.

\begin{equation*}
  \mathbf{a}_i = \begin{pmatrix} a_{i1} \\ ... \\ a_{in} \end{pmatrix}
\end{equation*}

\pagebreak

\section{Vector Operations}

For \(\mathbf{a}^\intercal = \begin{pmatrix} a_1 & ... & a_n \end{pmatrix}\) and \(\mathbf{b}^\intercal = \begin{pmatrix} b_1 & ... & b_n \end{pmatrix}\), \(\odot\) depicts the elementwise multiplication of two vectors with the same dimensionality.

\begin{equation*}
  \mathbf{a} \odot \mathbf{b} = \begin{pmatrix} a_1 * b_1 \\ ... \\ a_n * b_n \end{pmatrix}
\end{equation*}

The concatenation of vectors is abbreviated as follows:

\begin{equation*}
  \begin{pmatrix} \mathbf{a} \\ \mathbf{b} \end{pmatrix} = \begin{pmatrix} a_1 \\ ... \\ a_n \\ b_1 \\ ... \\ b_n \end{pmatrix}
\end{equation*}

All functions such as \(\tanh\) or \(\sigma\) are applied to a vector elementwise unless specified differently.

\begin{equation*}
  \tanh \begin{pmatrix} a_1 \\ ... \\ a_n \end{pmatrix} = \begin{pmatrix} \tanh(a_1) \\ ... \\ \tanh(a_n) \end{pmatrix}
\end{equation*}
